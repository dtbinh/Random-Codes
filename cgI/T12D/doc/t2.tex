\documentclass[12pt]{article}
\usepackage[brazilian]{babel}
\usepackage[utf8]{inputenc}
\usepackage[T1]{fontenc}

\sloppy

\title{Computação Gráfica I\\ Trabalho I}

\author{Matthias Oliveira de Nunes}

\begin{document}

\maketitle

\begin{abstract}

Este artigo descreve um relatório sobre o primeiro trabalho da diciplina de
computação gráfica I.

\end{abstract}

\section{Introdução}

O nosso trabalho consiste em um jogo de plataforma 2D, feito em C++ junto com
openGL. O objetivo do jogo é encostar no triângulo que existe no final da fase,
e para isso deve-se pular um buraco e desviar de um inimigo. O mais importante
no trabalho é o nosso projeto de classes e é isso que iremos ver agora.

\section{Projeto de Classes}

A principal classe do projeto é a GameObject. Ela define tudo que um objeto
precisa para poder se movimentar na tela. O personagem principal, o inimigo, a
plataforma móvel e o triângulo de final de fase herdam de GameObject, com isso
todos eles já tem tudo que necessitam para serem desenhados na tela e
manipulados, caso necessário.

\section{Posicionamento}

Todo GameObject é desenhado com o seu centro na origem e depois é transladado
para a sua devida posição. Essa translação vai ser o ponto exato em que o centro
do objeto se encontra no universo, por exemplo: Se vamos transladar um objeto 3
unidades em $x$, e 4 unidades em $y$, o centro do objeto vai estar no ponto (3,
4) no universo.

\section{Colisões}

Todo GameObject tem uma BoxCollider, que simplesmente são 4 pontos que definem a
área que o objeto ocupa no universo. Para detectar as colisões, são retirados
desse Collider o menor e maior valor de $x$, e o maior e menor valor de $y$.
Esses valores são utilizados na hora de comparar se o objeto está em uma certa
área do mapa, ou se já encostou no chão. Todas as colisões são baseadas no
posicionamento do objeto.

\section{Conclusão}

Aplicando os conhecimentos adquiridos na diciplina de Computação Gráfica I,
conseguimos criar um jogo simples e funcional, utilizando somente operações
básicas de matrizes e detecção de posição.

\end{document}
